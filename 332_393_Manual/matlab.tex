\chapter{Matlab}\label{chap:MATLAB}

The students of this course should be familiar with the basic ideas of
computer programming and Matlab from first year courses.

\section{Defining variables}

\begin{itemize}
    \item \verb|x=3|\\
          Defines the variable \verb|x| to be the constant 3.
    \item \verb|x=(1,2;3,4;5,6)|\\
          Defines the variable \verb|x| to be the \(3 \times 2\) matrix
          \begin{equation*}
              \begin{bmatrix}1&2\\3&4\\5&6\end{bmatrix}
          \end{equation*}
          Elements of a row are delimited by commas (or spaces) and each row is
          delimited by a semicolon.
\end{itemize}

\section{General Commands}

\begin{itemize}
    \item \verb|dir|\\
          Displays a list of files in the current directory.
    \item  \verb|open lab_1.mdl|\\
          Opens the specific file in the argument. We will be dealing mostly with
          \verb|*.mdl| and \verb|*.m| files.
    \item \verb|who|\\
          Displays a list of variables in the memory of \textsf{Matlab}.
    \item \verb|simulink|\\
          Opens the \verb|Simulink Browser Library| window.  Using the GUI control, you
          can drag and drop the blocks to build the \textsf{Simulink} models needed for
          each lab.  Details on using \textsf{Simulink} and WinCon are discussed in
          Appendix~\ref{chap:simulink}.
\end{itemize}

\section{Plotting}

\begin{itemize}
    \item \verb|plot (lab_1_Tachometer)|\\
          Plots the data from the variable \verb|lab_1_Tachometer| in \textsf{Matlab} memory.
          You can check the list of variables in the \textsf{Matlab} memory by using the
          \verb|who| command.

    \item \verb|plot (lab_1_Tachometer,'r:')|\\
          Plots the result in the memory of \textsf{Matlab} and specifies the colour of
          the graph to be red and the line style to be dotted.  Colour and line format
          are optional commands and they do not have to be specified for the plot
          command to produce an output. You can also specify one style parameter
          without the other.  The default colour is blue and the default line style is
          solid.  Table~\ref{tab:colour} is a list of colours and line styles that can
          be used with the plot command.
          \begin{table}[htbp]
              \centering
              \begin{tabular}{c c}
                  Line style/colour & \textsf{Matlab} command                                \\\toprule
                  solid             & '\verb|-|'                              \\
                  dashed            & '\verb|--|'                              \\
                  dotted            & '\verb|:|'                              \\
                  dash-dot          & '\verb|-.|'                              \\
                  blue              & '\verb|b|' or '\verb|blue|' \\
                  black             & '\verb|k|' or '\verb|black|' \\
                  cyan              & '\verb|c|' or '\verb|cyan|' \\
                  green             & '\verb|g|' or '\verb|green|' \\
                  magenta           & '\verb|m|' or '\verb|magenta|' \\
                  red               & '\verb|r|' or '\verb|red|' \\
                  white             & '\verb|w|' or '\verb|white|' \\
                  yellow            & '\verb|y|' or '\verb|yellow|' \\
              \end{tabular}
              \caption{Colour commands in \textsf{Matlab}}\label{tab:colour}
          \end{table}

    \item \verb|title ('Angular Velocity of the Motor')|\\
          Sets the title of the plot to the text in quotations.

    \item \verb|xlabel ('Time (s)')|\\
          Sets the x-axis label of the plot to the text in quotations.

    \item \verb|ylabel ('rad/sec')|\\
          Sets the y-axis label of the plot to the text in quotations.

    \item \verb|hold|\\
          Hold the current graph in figure and allow the user to plot more than one set
          of data on the same figure.

    \item \verb|hold off|\\
          Release the current graph in figure and allow a new plot to replace the
          current graph.
\end{itemize}

\section{Control System Toolbox}

\begin{itemize}
    \item \verb|sys = ss(A,b,c,D)|\\
          Defines the the state-space system from matrices \(\mat{A}\), \(\vect{b}\),
          \(\vect{c}\), and \(\vect{D}\).  For a model with \(n\) states and \(1\) output,
          \begin{itemize}
              \item \(\mat{A}\) is an \(n\times n\) matrix,
              \item \(\vect{b}\) is an \(n\times 1\) matrix,
              \item \(\vect{c}\) is a \(1\times n\) matrix (\(\vect{c}^{t}\) in our notation),
                    and
              \item \(\mat{D}=[0]\) (always true for this class).
          \end{itemize}


    \item \verb|h = tf([1 0],[1 2 10])|\\
          Defines the variable \verb|h| to be the transfer function
          \begin{equation*}
              \frac{s}{s^{2}+2s+10}.
          \end{equation*}

    \item \verb|h = zpk([1 0],[-1 -2 -10],[3])|\\
          Defines the variable \verb|h| to be the transfer function using the location
          of the zeros, poles, and a multiplicative constant:
          \begin{equation*}
              \frac{3s(s-1)}{(s+1)(s+2)(s+10)}.
          \end{equation*}

    \item \verb|h = tf(sys)|\\
          Defines the variable \verb|h| to be the transfer function for a given
          state-space system.

    \item \verb|sys = tf2ss[tf]|\\
          Gives the SISO linear system corresponding to the transfer function \verb|tf|.

    \item \verb|bode(sys)|\\
          Produces the Bode plots for the given system.

    \item \verb|impulse(sys)|\\
          Produces the impulse response for the given system.

    \item \verb|nyquist(sys)|\\
          Produces the Nyquist plot for the given system.

    \item \verb|margin(sys)|\\
          Produces the gain and phase margins with associated crossover frequencies.
\end{itemize}

%%% Local Variables: 
%%% mode: latex
%%% TeX-master: "lab-manual"
%%% End: 
