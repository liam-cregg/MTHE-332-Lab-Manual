\chapter{State Space Representations}\label{chap:EOMs}

\section{Rotary Pendulum}
To describe the configuration of the physical system relative to some reference configuration, we use coordinates (not to be confused with the coordinate axes shown in Figure~\ref{fig:lab1_rotary_flexible_beam} - we're talking about variables $\theta$ and $\alpha$ here) which we call the \emph{generalized coordinates} of the system, and we denote the $i^{th}$ generalized coordinate by $q_i$. We define the kinetic and potential energies of the system by functions of the generalized coordinates and their time derivatives, and denote them by $T$ and $V$, respectively. To summarize the dynamics of our physical system, we use the function $L$ (called the \emph{Lagrangian} of the dynamical system) which is defined by
\[
    L=T-V.
\]
In modelling a physical system, we must also consider external forces applied to the system. We do this by defining a \emph{generalized force} for each generalized coordinate $q_i$ which takes into account the effects of externally applied forces (or torques if $q_i$ is an angle) that are excluded from $V$ (e.g., the force of gravity is a potential force, not a generalized force). We denote the generalized force for $q_i$ by $Q_{q_i}$.
The Euler-Lagrange equation is then given as follows:
\begin{equation}\label{equation:lab1_lagrange_equns}
    \frac{d}{dt} \Bigg(\frac{\partial L}{\partial \dot{q_i}}\Bigg)  - \frac{\partial L}{\partial q_i} = Q_{q_i}.
\end{equation}
To find the equation of motion for a particular coordinate $q_i$, all one must do is find $L$, $V$ and then evaluate~\eqref{equation:lab1_lagrange_equns}. For the rotary pendulum system, our coordinates will be the rotary shaft angle, $\theta$, and the pendulum rod angle, $\alpha$, so that our coordinate vector is $q(t)^T=[\theta(t) \; \alpha(t)]$. See Figure~\ref{fig:lab2_rotary_pendulum} for what these coordinates look like in the physical system.

We first wish to write down the EOMs for the rotary pendulum system (for coordinates \( \theta \) and \( \alpha \)) using~\eqref{equation:lab1_lagrange_equns}. Since the rotary shaft is actuated and the pendulum rod is not, and accounting for viscous friction, one can infer that $Q_\theta = \tau - B_r \dot{\theta} $ and $Q_\alpha = -B_p \dot{\alpha}$, where $\tau$ is the torque applied by the servo motor, and $B_r$, $B_p$ are the viscous damping coefficients for the rotary shaft and pendulum rod, respectively. Deriving $T$ and $V$ for the Lagrangian is not in the scope of this course, so they are provided below:
\begin{equation*}
    \begin{cases}
        T = \left(\frac{1}{2} m_p L_{r}^2 + \frac{1}{8} m_p L_{p}^2 - \frac{1}{8} m_p L_{p}^2 \cos^2(\alpha) + \frac{1}{2} J_r\right) \dot{\theta}^2 + \left(\frac{1}{2} J_p + \frac{1}{8} m_p L_{p}^2 \right) \dot{\alpha}^2 - \frac{1}{2} m_p L_p L_r \cos(\alpha) \dot{\theta} \dot{\alpha} \\
        V = \frac{1}{2} m_p L_p g \cos(\alpha)
    \end{cases}
\end{equation*}
where $J_p$, $m_p$, and $L_p$ are the moment of inertia about the centre of mass, mass, and length of the pendulum, respectively; $J_r$, $m_r$, and $L_r$ are the moment of inertia about the centre of mass, mass, and length of the rotary shaft, respectively; and $g$ is the acceleration due to gravity (on earth).
We compute the following sets of equations:
\[
    \begin{cases}
        \frac{\partial{L}}{\partial \theta}=0                                                                                                                                                      \\
        \frac{\partial L}{\partial \dot{\theta}}=\left( m_pL_r^2 +\frac{1}{4} m_p L_p^2-\frac{1}{4}m_pL_p^2\cos^2(\alpha)+J_r\right)\dot{\theta} - \frac{1}{2} m_pL_pL_r\cos{(\alpha)}\dot{\alpha} \\
        \frac{d}{dt} \left(\frac{\partial L}{\partial \dot{\theta}}\right)= \left(m_pL_r^2 +\frac{1}{4}m_pL_p^2-\frac{1}
        {4}m_pL_p^2\cos^2(\alpha)+J_r\right)\ddot{\theta} + \frac{1}{2}m_pL_p^2\sin{(\alpha)}\cos{(\alpha)} \dot{\alpha}\dot{\theta}                                                               \\+ \frac{1}{2}m_pL_pL_r\sin{(\alpha)}\dot{\alpha}^2-\frac{1}{2}m_pL_pL_r\cos{(\alpha)}\ddot{\alpha} \\
    \end{cases}
\]
\[
    \begin{cases}
        \frac{\partial L}{\partial \alpha}=\frac{1}{4}m_pL_p^2\cos{(\alpha)}\sin{(\alpha)}\dot{\theta}^2+\frac{1}{2}m_pL_pL_r\sin{(\alpha)}\dot{\theta}\dot{\alpha}+ \frac{1}{2}m_pL_pg\sin{\alpha} \\
        \frac{\partial L}{\partial \dot{\alpha}}= \left(J_p+\frac{1}{4}m_pL_p^2\right)\dot{\alpha}-\frac{1}{2}m_pL_pL_r\cos{(\alpha)}\dot{\theta}                                                   \\
        \frac{d}{dt} \left(\frac{\partial L}{\partial \dot{\alpha}}\right)= \left(J_p+\frac{1}{4}m_pL_p^2\right)\ddot{\alpha}+\frac{1}{2}m_pL_pL_r\sin{(\alpha)}\dot{\theta}\dot{\alpha}-\frac{1}{2}m_pL_pL_r\cos{(\alpha)} \ddot{\theta}
    \end{cases}
\]
Thus, via the Euler-Lagrange equation we get the following EOMs for the coordinates. For \( \theta \):
\begin{align*}
     & \left(m_p L_{r}^{2} + \frac{1}{4} m_p L_{p}^{2} - \frac{1}{4} m_p L_{p}^{2} \cos^2(\alpha) + J_r\right) \ddot{\theta} - \frac{1}{2} m_p L_p L_r \cos(\alpha) \ddot{\alpha} \\
     & + \frac{1}{2} m_p L_{p}^{2} \sin(\alpha)\cos(\alpha) \dot{\theta}\dot{\alpha} + \frac{1}{2}m_p L_p L_r \sin(\alpha) \dot{\alpha}^{2} = \tau - B_r \dot{\theta}
\end{align*}
And for \( \alpha \):

\begin{align*}
     & -\frac{1}{2} m_p L_p L_r \cos(\alpha) \ddot{\theta} + \left(J_p + \frac{1}{4} m_p L_{p}^{2}\right)\ddot{\alpha} - \frac{1}{4} m_p L_{p}^{2} \sin(\alpha)\cos(\alpha) \dot{\theta}^{2} \\
     & - \frac{1}{2} m_p L_{p} g \sin(\alpha) = - B_p \dot{\alpha}
\end{align*}

\subsection{Linearization}
Note that these EOMs are nonlinear, so we wish to linearize these equations about an equilibrium point. Recall that to linearize a multivariate function \( f \) of variables \( z^T = [\theta \; \alpha \; \dot{\theta} \; \dot{\alpha} \; \ddot{\theta} \; \ddot{\alpha}] \) around an equilibrium point \( z_{0}^T = [\theta_0 \; \alpha_0 \; \dot{\theta}_0 \; \dot{\alpha}_0 \; \ddot{\theta}_0 \; \ddot{\alpha}_0] \), we use
\[
    f_\text{lin} = f(z_0) + \left(\frac{\partial f(z)}{\partial \theta}\right) \bigg|_{z_0}  (\theta - \theta_0) +  \left(\frac{\partial f(z)}{\partial \alpha}\right) \bigg|_{z_0}  (\alpha - \alpha_0) + \dots +  \left(\frac{\partial f(z)}{\partial \ddot{\alpha}}\right) \bigg|_{z_0}  (\ddot{\alpha} - \ddot{\alpha}_0)
\]
We consider two cases for the equilibrium point.

\subsubsection*{Downwards Position (\( \theta_0=0, \alpha_0=\pi \))}
For the generalized coordinate \( \theta \), we compute
\[
    \left(m_p L_{r}^{2} + J_r\right) \ddot{\theta} + \frac{1}{2} m_p L_p L_r \ddot{\alpha} = \tau - B_r \dot{\theta}
\]
And for the generalized coordinate \( \alpha \), we compute
\[
    \frac{1}{2} m_p L_p L_r \ddot{\theta} + \left(J_p + \frac{1}{4} m_p L_{p}^{2}\right)\ddot{\alpha} + \frac{1}{2} m_p L_{p} g \alpha = - B_p \dot{\alpha}
\]

\subsubsection*{Inverted Position (\( \theta_0=0, \alpha_0=0 \))}
For the generalized coordinate \( \theta \), we compute
\[
    \left(m_p L_{r}^{2} + J_r\right) \ddot{\theta} - \frac{1}{2} m_p L_p L_r \ddot{\alpha} = \tau - B_r \dot{\theta}
\]
And for the generalized coordinate \( \alpha \), we compute
\[
    -\frac{1}{2} m_p L_p L_r \ddot{\theta} + \left(J_p + \frac{1}{4} m_p L_{p}^{2}\right)\ddot{\alpha} - \frac{1}{2} m_p L_{p} g \alpha = - B_p \dot{\alpha}
\]

\subsection{State-Space Representation}
In order to find the state space representation, we first write the linearized EOMs in matrix form, i.e.,
\[
    \left[\begin{array}{c c}
            e_{11} & e_{12} \\
            e_{21} & e_{22}
        \end{array}\right]
    \left[\begin{array}{c}
            \ddot{q}_{1} \\
            \ddot{q}_{2}
        \end{array}\right] +
    \left[\begin{array}{c c}
            f_{11} & f_{12} \\
            f_{21} & f_{22}
        \end{array}\right]
    \left[\begin{array}{c}
            \dot{q}_{1} \\
            \dot{q}_{2}
        \end{array}\right] +
    \left[\begin{array}{c}
            g_1 \\
            g_2
        \end{array}\right] =
    \left[\begin{array}{c}
            \tau_1 \\
            \tau_2
        \end{array}\right],
\]
and group all non double-derivative terms to the right. We can now explicitly solve for $\left[\begin{array}{c}
            \ddot{\theta} \\
            \ddot{\alpha}
        \end{array}\right]$ and put it in a relatively compact form. Letting our state be
\[
    \mathbf{x}(t) =
    \left[\begin{array}{c}
            \theta(t)       \\
            \alpha(t)       \\
            \dot{\theta}(t) \\
            \dot{\alpha}(t)
        \end{array}\right]
\]
we obtain the following for each of the equilibirum points.

\subsubsection*{Downwards Position (\( \theta_0=0, \alpha_0=\pi \))}
\begin{align*}
     & \left[\begin{array}{c}
            \dot{\theta}(t)  \\
            \dot{\alpha}(t)  \\
            \ddot{\theta}(t) \\
            \ddot{\alpha}(t)
        \end{array}\right] = \frac{1}{J_T}
    \left[\begin{array}{c c c c}
            0 & 0                                                     & J_T                                            & 0                                  \\
            0 & 0                                                     & 0                                              & J_T                                \\
            0 & \frac{1}{4} m_{p}^2 L_{p}^2 L_r g                     & -\left(J_p + \frac{1}{4} m_p L_{p}^2\right)B_r & \frac{1}{2} m_p L_p L_r B_p        \\
            0 & -\frac{1}{2} m_p L_p g \left(J_r + m_p L_{r}^2\right) & \frac{1}{2} m_p L_p L_r B_r                    & -\left(J_r + m_p L_{r}^2\right)B_p
        \end{array}\right]
    \left[\begin{array}{c}
            \theta(t)       \\
            \alpha(t)       \\
            \dot{\theta}(t) \\
            \dot{\alpha}(t)
        \end{array}\right]                    \\
     & + \frac{1}{J_T}
    \left[\begin{array}{c}
            0                             \\
            0                             \\
            J_p + \frac{1}{4} m_p L_{p}^2 \\
            -\frac{1}{2} m_p L_p L_r
        \end{array}\right] \tau
\end{align*}

\subsubsection*{Inverted Position (\( \theta_0=0, \alpha_0=0 \))}
\begin{align*}
     & \left[\begin{array}{c}
            \dot{\theta}(t)  \\
            \dot{\alpha}(t)  \\
            \ddot{\theta}(t) \\
            \ddot{\alpha}(t)
        \end{array}\right] = \frac{1}{J_T}
    \left[\begin{array}{c c c c}
            0 & 0                                                    & J_T                                            & 0                                  \\
            0 & 0                                                    & 0                                              & J_T                                \\
            0 & \frac{1}{4} m_{p}^2 L_{p}^2 L_r g                    & -\left(J_p + \frac{1}{4} m_p L_{p}^2\right)B_r & -\frac{1}{2} m_p L_p L_r B_p       \\
            0 & \frac{1}{2} m_p L_p g \left(J_r + m_p L_{r}^2\right) & -\frac{1}{2} m_p L_p L_r B_r                   & -\left(J_r + m_p L_{r}^2\right)B_p
        \end{array}\right]
    \left[\begin{array}{c}
            \theta(t)       \\
            \alpha(t)       \\
            \dot{\theta}(t) \\
            \dot{\alpha}(t)
        \end{array}\right]                    \\
     & + \frac{1}{J_T}
    \left[\begin{array}{c}
            0                             \\
            0                             \\
            J_p + \frac{1}{4} m_p L_{p}^2 \\
            \frac{1}{2} m_p L_p L_r
        \end{array}\right] \tau
\end{align*}
Evaluating the symbolic state-space matrices $A$ and $B$ using the following values:
\[
    \begin{cases}
        J_p = 0.001199 \; kg \cdot m^2     \\
        J_r = 0.000998 \; kg \cdot m^2     \\
        B_p = 0.0024 \; \frac{N\cdot s}{m} \\
        B_r = 0.0024 \; \frac{N\cdot s}{m} \\
        L_p = 0.3365 \; m                  \\
        L_r = 0.2159 \; m                  \\
        m_p = 0.1270 \; kg                 \\
        m_r = 0.2570 \; m.
    \end{cases}
\]

we get
\subsubsection*{Downwards Position (\( \theta_0=0, \alpha_0=\pi \))}
\[
    A =
    \left[\begin{array}{c c c c}
            0 & 0       & 1      & 0       \\
            0 & 0       & 0      & 1       \\
            0 & 81.38   & -93.49 & 0.0038  \\
            0 & -122.03 & 89.97  & -0.0058 \\
        \end{array}\right]
\]

\[
    B =
    \left[\begin{array}{c}
            0     \\
            0     \\
            83.64 \\
            -80.48
        \end{array}\right].
\]

\subsubsection*{Inverted Position (\( \theta_0=0, \alpha_0=0 \))}
\[
    A =
    \left[\begin{array}{c c c c}
            0 & 0      & 1      & 0     \\
            0 & 0      & 0      & 1     \\
            0 & 81.38  & -46.05 & -0.93 \\
            0 & 122.03 & -44.31 & -1.39
        \end{array}\right],
\]
\[
    B =
    \left[\begin{array}{c}
            0     \\
            0     \\
            83.64 \\
            80.48
        \end{array}\right],
\]

For both positions, given that the physical system’s sensors are limited to reading the rotor angle and the deflection angle, our state space matrices \(C\) and \(D\) are

\[
    C =
    \left[\begin{array}{c c c c}
            1 & 0 & 0 & 0 \\
            0 & 1 & 0 & 0
        \end{array}\right]
\]
and
\[
    D =
    \left[\begin{array}{c}
            0 \\
            0
        \end{array}\right].
\]

\section{Flexible Beam}
\begin{equation*}
    \begin{cases}
        T = \frac{1}{2}J_r \dot{\theta}^2 + \frac{1}{2} J_b \left(\dot{\theta}+\dot{\alpha}\right)^2 \\
        V = \frac{1}{2} K_b \alpha^2
    \end{cases}
\end{equation*}

\[
    \begin{cases}
        \frac{\partial L}{\partial \theta}=0                                                                                             \\
        \frac{\partial L}{\partial \dot{\theta}}=J_r\dot{\theta}+J_b\left(\dot{\theta}+\dot{\alpha}\right)                               \\
        \frac{d}{dt} \left(\partial \frac{L}{\partial \dot{\theta}}\right)= J_r\ddot{\theta}+J_b\left(\ddot{\theta}+\ddot{\alpha}\right) \\
    \end{cases}
\]

\[
    \begin{cases}
        \frac{\partial L}{\partial \alpha}=K_b\alpha                                                                    \\
        \frac{\partial L}{\partial \dot{\theta}}=J_b\left(\dot{\theta}+\dot{\alpha}\right)                              \\
        \frac{d}{dt} \left(\frac{\partial L}{\partial \dot{\theta}}\right)= J_b\left(\ddot{\theta}+\ddot{\alpha}\right) \\
    \end{cases}
\]

which yield, via the Euler-Lagrange equation,

\[
    \left(J_r + J_b\right)\ddot{\theta} + J_b \ddot{\alpha} = \tau - B_r \dot{\theta}.
\]

\[
    J_b \left(\ddot{\theta} + \ddot{\alpha}\right) + K_b \alpha = -B_b \dot{\alpha}.
\]

Thus, letting our state be
\[
    x(t) =
    \left[\begin{array}{c}
            \theta(t)       \\
            \alpha(t)       \\
            \dot{\theta}(t) \\
            \dot{\alpha}(t)
        \end{array}\right].
\]
and assuming the viscous damping of the beam is negligible (so $B_b = 0$), we get the following state-space representation:

\[
    \left[\begin{array}{c}
            \dot{\theta}(t)  \\
            \dot{\alpha}(t)  \\
            \ddot{\theta}(t) \\
            \ddot{\alpha}(t)
        \end{array}\right] =
    \left[\begin{array}{c c c c}
            0 & 0                                          & 1                & 0 \\
            0 & 0                                          & 0                & 1 \\
            0 & \frac{K_b}{J_r}                            & -\frac{B_r}{J_r} & 0 \\
            0 & -\frac{K_b\left(J_b + J_r\right)}{J_b J_r} & \frac{B_r}{J_r}  & 0
        \end{array}\right]
    \left[\begin{array}{c}
            \theta(t)       \\
            \alpha(t)       \\
            \dot{\theta}(t) \\
            \dot{\alpha}(t)
        \end{array}\right] \\ +
    \left[\begin{array}{c}
            0             \\
            0             \\
            \frac{1}{J_r} \\
            -\frac{1}{J_r}
        \end{array}\right] \tau
\]

Evaluating these matrices numerically, we get

\[
    A =
    \left[\begin{array}{c c c c}
            0 & 0       & 1      & 0 \\
            0 & 0       & 0      & 1 \\
            0 & 623.77  & -40.49 & 0 \\
            0 & -965.53 & 40.49  & 0
        \end{array}\right]
\]
and
\[
    B =
    \left[\begin{array}{c}
            0      \\
            0      \\
            61.775 \\
            -61.775
        \end{array}\right].
\]

And using the fact that \(y = Cx + Du\),

\[
    C =
    \left[\begin{array}{c c c c}
            1 & 0 & 0 & 0 \\
            0 & 1 & 0 & 0
        \end{array}\right]
\]
and
\[
    D =
    \left[\begin{array}{c}
            0 \\
            0
        \end{array}\right],
\]