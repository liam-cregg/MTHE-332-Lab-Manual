\chapter{The Nyquist criterion}

In this lab, you will use the Nyquist contour to determine if a system is
IBIBO stable.  The Nyquist plot is a conformal mapping of a closed loop in
complex plane and it is a tool for the understanding of the stability of a
closed-looped system.  Furthermore, you will be examining the relationship
between the Nyquist contour and the Bode plot, thus familiarizing yourself
with such terms as \emph{crossover frequency}\@, \emph{gain margin}\@, and
\emph{phase margin}\@.

\section{Prelab}

Before you go into the lab, you should read the following:
\begin{itemize}
    \item Sections 7.1.2 and 7.2 from the course text on the Nyquist criterion.
\end{itemize}

\section{Procedure}

\subsection{Using the Bode plot to sketch the Nyquist
    contour}\label{sec:bodeNyquist}

For this part of the lab, we will be using the motor in the unity gain
feedback configuration shown in Figure~\ref{fig:generic}\@.
\begin{figure}[htbp]
    \centering
    \begin{picture}(230,75)
        \put(0,50){\(\hat\theta_{d}(s)\)}
        \put(25,53){\vector(1,0){15}}
        \put(45,53){\circle{7}}
        \put(32,57){+}
        \put(50,53){\vector(1,0){20}}
        \put(73,37){\framebox(35,33){\(R_{C}(s)\)}}
        \put(110,53){\vector(1,0){40}}
        \put(152,37){\framebox(35,33){\(R_{P}(s)\)}}
        \put(190,53){\vector(1,0){30}}
        \put(224,50){\(\hat\theta(s)\)}
        \put(206,53){\line(0,-1){45}}
        \put(205,8){\line(-1,0){160}}
        \put(45,8){\vector(0,1){40}}
        \put(35,48){\line (1,0) {4}}
    \end{picture}
    \caption{Feedback system with disturbance}\label{fig:generic}
\end{figure}%
This setup should be quite familiar to you by now, so you should have a
well-developed intuition concerning the IBIBO stability of such a system.  We
will be taking the output of the system to be the motor velocity, \(\omega \),
so our plant transfer function is \(R_{P}(s)=\frac{k_{E}}{s+\frac{1}{\tau}}\).
In Lab 2 you constructed the Bode plot for this system
experimentally.
\begin{enumerate}
    \item Using the Bode plot from Lab 2, determine the gain
          and phase crossover frequencies, and the phase and gain margins at these
          frequencies; see Section 7.2.2 from the course text.

    \item Using the Bode plot determine the general transfer function, from the
          transfer function sketch the corresponding Nyquist contour.  Looking also at
          the transfer function, determine whether or not the system is IBIBO stable.

    \item Start up \textsf{Matlab} and define \verb|t| to be the transfer
          function of the open-loop motor scheme.  Essentially, we are just setting
          \(R_{C}(s)=1\) to see how motor behaves on its own.

    \item To plot the Nyquist contour, use the \verb|Nyquist| command in
          \textsf{Matlab}.

    \item Is the Nyquist contour consistent with what you know about the IBIBO
          stability of the system?  Comment on both the poles of the transfer function,
          and the zeroes of the characteristic polynomial.
\end{enumerate}

\subsection{Using a proportional controller}

We will now turn our attention to the task of using the Nyquist contour as a
tool in determining what range of values of a proportional controller can
take on in order to make the system IBIBO stable.

The plant transfer function \(R_{C}(s) = K\) will be used in the feedback
configuration depicted in Figure~\ref{fig:generic}\@.  The transfer function
for this system is simply
\begin{equation*}
    T(s)=\frac{KR_{P}(s)}{1+KR_{P}(s)}.
\end{equation*}
What is important to note is that the condition \(KR_{P}(s)=-1\) is equivalent
to the condition \(R_{P}=-\frac{1}{K}\). This seems obvious in itself, but what
this allows you to do is to use the Nyquist contour for \(R_{P}(s)\) on its own
(i.e., with \(K=1\)), and determine where you would like the point
\(\frac{1}{K}\) to be. This allows you to get a lot of mileage out of only one
plot. The alternative is to construct the Nyquist contour for many values of
\(K\), and see if each contour has the required number of encirclements of the
point \(-1+\imag0\).  This may not seem like a bad alternative if you have
access to a computer, but the method of normalizing the proportional gain is
much more efficient.

\begin{enumerate}
    \item In \textsf{Matlab}, define the plant transfer function to be
          \(R_{P}=\frac{s+1}{s(\frac{s}{10}-1)}\).

    \item Produce both the Bode plot and the Nyquist contour.  Take a minute to
          further reaffirm your faith in the relationship between the two.

    \item What are the crossover frequencies, and what are the gain and phase
          margins at these frequencies?

    \item Since our plant transfer function has one pole in the right-half
          complex plane, we require one counterclockwise encirclements of the point
          \(\frac{1}{K}\). Using this information, in which region should we place the
          point \(-\frac{1}{K}\)?

    \item What values of \(K\) satisfy the above requirement?

    \item Repeat the above procedure for the plant transfer function
          \(R_{P}(s)=\frac{1}{s{(s+1)}^{2}}\).  Remember, Nyquist plots always close
          clockwise.
\end{enumerate}

\subsection{The phase margin}

In this part of the lab we will quickly demonstrate the phase margin as it
applies to the Nyquist contour.
\begin{enumerate}
    \item We will use the plant transfer function
          \(R_{P}(s)=\frac{1}{s{(s+1)}^{2}}\).  Produce the Bode plot and the Nyquist
          contour of this system.

    \item Determine how much you have to boost the gain (in decibels) to provide
          a phase margin of \(0^{\circ}\) at a gain cross over frequency of
          \(10^{n}\frac{\textup{rad}}{\textup{sec}}\), \(n\in\mathbb{Z}\).  Will the %chktex 35
          resulting system, with this gain boosted by using a proportional controller,
          be IBIBO stable?

    \item The number you get is a scalar that represents the factor by which you
          can ``stretch'' (or ``shrink'' as the case may be) the Nyquist contour
          towards the point \(-1+\imag0\) and still maintain IBIBO stability (assuming
          the system is stable for \(K=1\), which our system is).  Looking at the Nyquist
          contour, does this scalar quantity seem to make sense?  Is this consistent
          with your work from Section~\ref{sec:bodeNyquist}?
\end{enumerate}

When you have completed the lab, make sure you save your files in the folder
you created in Lab~\ref{chap:intro}\@.

%%% Local Variables: 
%%% mode: latex
%%% TeX-master: "lab-manual"
%%% End: 
