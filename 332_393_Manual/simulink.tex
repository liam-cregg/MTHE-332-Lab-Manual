\chapter{Simulink}\label{chap:simulink}

\textsf{Simulink} allows simulation of complex control systems using a drag
and drop block diagram interface.  \textsf{Simulink} is especially useful
when used in conjunction with the Real-Time Workshop which allows
\textsf{Simulink} diagrams to be converted into C codes which can be run in
real-time on a number of so-called targets (the PC being one such target).

\section{Starting}

\begin{itemize}
    \item To use \textsf{Simulink}, one must first start \textsf{Matlab}.  After
          starting \textsf{Matlab}, you would type \verb|simulink| in the
          \textsf{Matlab} command prompt to get the \verb|Simulink Library Browser|
          window.

    \item Selecting \verb|File|\(\to \)\verb|New|\(\to \)\verb|Model| (or
          \verb|Ctrl+N|) while in the \verb|Simulink Library Browser| will give you a
          blank model window into which you can drag-and-drop system blocks from the
          \verb|Simulink Library Browser| to build a \textsf{Simulink} model.  These
          models can be saved as \verb|*.mdl| files for future simulations and editing.
\end{itemize}

\section{Building a \textsf{Simulink} model}

\begin{itemize}
    \item To connect the output of block~A to the input of block~B, simply
          left-click the output port of block~A and drag the line that would appear
          into the input port of the block~B.

    \item In order to connect the output of a block into the inputs of multiple
          blocks at the same time, you can right-click on an existing connection to get
          another line and drag that connection into the input of another block.

    \item When a block is dragged into the model window, it will be given a
          generic name.  For example, when the \verb|scope| block is dragged into a
          model, it would simply be labelled as ``scope'' and if it was the second
          \verb|scope| block to be dragged into the model, it would be labelled as
          ``scope2''.  It is a good practice to rename these blocks and give them more
          appropriate labels.  These names are usually suggested in the diagram of the
          models in each lab. To rename a block, simply click on the existing name once
          and edit.
\end{itemize}

\section{Simulations}

\begin{itemize}
    \item One could view and edit the simulation parameters by clicking on
          \begin{center}
              \verb|Simulation|\(\to \)\verb|Simulation Parameters|
          \end{center}
          (or \verb|Ctrl+E|) while editing the model.  For the purpose of this course,
          there are only three things that you have to worry about:
          \begin{enumerate}
              \item Start and stop time: The default start time is 0 and the default stop
                    time is 10.  Usually, there is no reason to change the default start time,
                    but you might find it useful to extend the stop time so that you could
                    observe the simulation for a longer period of time.

              \item Solver Method: You must have this set to a fixed-step when implementing
                    real-time controllers. The default solver is a discrete method. You need to
                    change it from the default method to \verb|ode4|, which is an implementation
                    of the Runge-Kutta method.

              \item Step size: The default setting of 0.001 corresponds to 1000 Hz sampling
                    frequency.  This is the fastest rate at which the system can sample.
          \end{enumerate}
\end{itemize}

\section{Plotting}

\begin{itemize}
    \item The outputs of a simulation can be captured by using the
          \verb|To Workspace| block.  The data would be recorded as a variable in
          memory of \textsf{Matlab}.  The default name of the output is \verb|simout|,
          but you should change the name of this output just as you would give
          appropriate label to the block itself.  One could plot the simulation output
          by using plot command discussed in Appendix~\ref{chap:MATLAB}\@.
\end{itemize}

\subsection{Saving data via the ``To Workspace'' block}

This is the preferred method of saving data to your \textsf{Matlab}
workspace.  The \verb|To Workspace| block can be found at
\begin{center}
    \verb|Simulink|\(\to \)\verb|Sinks|
\end{center}
Drag this block into your workspace and connect it to the variable you wish
to save.  Double click on the block to configure it.  Choose a good variable
name and in the \verb|save format| drop down menu select
\verb|Structure with time|.  After the simulation the data will be
automatically saved to your \textsf{Matlab} workspace and you can plot it
with the command
\begin{center}
    \verb|plot(varname.time,varname.signals.values)|
\end{center}
replacing ``\verb|varname|'' with the variable name you chose when
configuring the \verb|To Workspace| block.

\subsection{Saving data from a scope}

You can also save scope data, but the scope seems to have short-term memory
loss which makes it one of the most useless blocks in the simulink library.
Nevertheless if you need to save the data from a scope, follow these
instructions.
\begin{enumerate}
    \item Double click on the scope you wish to save the data from.
    \item Click on the Parameters Icon (in the top left of the scope dialog box).
    \item Data History Tab
          \begin{itemize}
              \item Uncheck box \verb|Limit data points|
              \item Check box \verb|Save data to workspace|
              \item Choose a variable name
              \item Select \verb|Array| from the Format drop down menu.
          \end{itemize}
    \item Click \verb|Apply| and \verb|OK| to exit.
\end{enumerate}
After the simulation has been performed, the data will appear as an array in
your \textsf{Matlab} workspace.  You can plot the data with the
command \begin{center}
    \verb|plot(ScopeData1(:,1), ScopeData1(:,2))|
\end{center}
where \verb|ScopeData1| is the variable name that you set in the above steps.

%%% Local Variables: 
%%% mode: latex
%%% TeX-master: "lab-manual"
%%% End: 
